\documentclass[12pt]{article}
\usepackage[fleqn]{amsmath}
\usepackage{amsfonts}
\usepackage[utf8]{inputenc}
\usepackage{amssymb}
\usepackage{listings}
\lstset{
  basicstyle=\small
}
\usepackage{color}
\usepackage[utf8]{inputenc} 
\usepackage[top=2cm, left=2.5cm]{geometry}
\title{Thermo 2 Formelsammlung}
\author{}

\usepackage{tcolorbox}

\begin{document}

\maketitle
\renewcommand{\arraystretch}{2}
\definecolor{backblue}{HTML}{D2E2FF}
\definecolor{borderblue}{HTML}{43587F}

\section{Bilanzen in Strömungen Strömungen}
\subsection{Masse}
\begin
{tcolorbox}[colback=backblue,colframe=borderblue,title=Massenbilanz]
Der Massenstrom durch eine Düse, Röhre o.Ä. muss stets konstant sein.
\begin{align*}
\dot{m}=\rho\;A\;c = const\\
\text{Differenzieren, dann Dividieren durch $\rho\; A\; c$}\\
\text{Diese Gleichung nennt man ugs. Konti-Gleichung}\\
\frac{d \rho}{\rho}+\frac{dA}{A}+\frac{dc}{c}=0
\end{align*}
\end{tcolorbox}
\subsection{Impuls}
Die Summe aller Kräfte, die auf die Masse wirken, verursachen eine Impulsänderung.
\begin{align*}
\sum F_p	&= ...=-Adp \\
&\text{Achtung, diese Fläche ist Fläche am Anfang des Rohres}\\
&\text{Dazu kommt Reibung}\\
\sum F_r&=-\tau \;U\;dx=-dF_r\;cos\;\alpha \\
&\text{Summe der beiden Kräfte für die Impulsbilanz}\\
\sum F_x&=-A\;dp -\tau \; U \; dx
\end{align*}
Für die Impulsbilanz wird die gesamte Masse im System betrachtet und die Änderung der Geschwindigkeit im System hinzugezogen:
\begin{align*}
c&=c(x), x=x(t) \\
\frac{dc}{dt}&=c\frac{dc}{dx}\\
dm&=\left(\rho A + \frac{d(\rho A)}{2} \right)dx = \rho A dx\\
&\text{Zusammen ergibt das die Impulsänderung}\\
dmc\frac{dc}{dt}&=\rho \; A\; dx \; c \frac{dc}{dx}
\end{align*}
\begin
{tcolorbox}[colback=backblue,colframe=borderblue,title=Impulsbilanz]
Wenn man die Impulsbilanz integriert, kommt man auf
\begin{align*}
&-A(p_2-p_1)-\tau U(x_2-x_1)=\rho A \frac{1}{2}(c_2^2-c_1^2)\\
&\text{Außerdem: Differentielle stationäre Impulsgleichung}\\
&-\frac{dp}{dx}-\frac{dr_x}{dx}=\rho \; c \frac{dc}{dx}\\
&\text{Ohne Reibung nennt man das die Euler-Gleichung}\\
&-\frac{dp}{dx}=\rho c \frac{dc}{dx} \rightarrow -\Delta p=\rho \frac{\Delta c^2}{2}\\
&\text{Die integrierte Form mit den Differenzen,}\\
&\text{nennt man die Bernoulli-Gleichung:}\\
&p+\frac{\rho}{2}c^2=const
\end{align*}
\end{tcolorbox}
\subsection{Energie}
Differentielle Energiegleichung, massebezogen
\begin{align*}
\delta q + \delta w^t = dh + \frac{1}{2}dc^2+gdz\\
\text{Adiabat, ohne technische Arbeit}\\
h_0-h=\frac{c^2-c_0^2}{2} \\
\text{Strömung aus der Ruhelage}\\
h_0-h=c_p(T_0-T)=\frac{c^2}{2}\\
\rightarrow c^2=2c_pT_0(1-\frac{T}{T_0})
\end{align*}
\subsection{WH: Isentrope Übergänge}
\begin{tcolorbox}[colback=backblue,colframe=borderblue,title=Isentropie]
\begin{align*}
\kappa&=\frac{c_p}{c_v}\\
\frac{c_p}{R}&=\frac{\kappa}{\kappa-1}\\
pV^\kappa&=const\\
\frac{T_2}{T_1}&=\left(\frac{p_2}{p_1}\right)^{\frac{\kappa-1}{\kappa}}\\
\frac{T_2}{T_1}&=\left(\frac{V_1}{V_2}\right)^{\kappa-1}
\end{align*}
\end{tcolorbox}
\subsection{Schall}
\begin{align*}
a^2=\left(\frac{\delta p}{\delta \rho}\right)_s\\
\frac{dp}{p}=\kappa \frac{d\rho}{\rho}\\
a=\sqrt{\kappa\frac{p}{\rho}}=\sqrt{\kappa R T}
\end{align*}
\end{document}